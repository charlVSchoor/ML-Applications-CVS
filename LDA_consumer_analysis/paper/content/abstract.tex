
Measuring hedonic and utilitarian consumption patterns has proven to be a difficult undertaking in marketing literature. However, the advent of machine learning and big data have brought about a new dimension of viewing marketing questions, such as the measurement and prediction of hedonic versus utilitarian consumer behaviour. The purpose of this paper is to consider a different approach to modeling hedonic and utilitarian behaviour by making use of text-mining, explanatory and predictive models. I use Latent Dirichlet Allocation to model product descriptions augmented with Wikipedia definitions into various topics. These topics are then used to explain and predict consumer behaviour by means of a proxy variable for utilitarian behaviour, namely gift purchases. The practical implication of this work is a more relational distribution of consumer purchases than would be obtained by considering products alone, which assist in having a distribution for individual consumers over a new product dimension. The results indicate that households that associate with stability purchases are more related to gift purchasing 




