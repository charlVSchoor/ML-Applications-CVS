\section{Theoretical Background} % (fold)
\label{sec:theoretical_background}

Defining utilitarian and hedonic consumption is historically been done by considering certain products as hedonic products \citep{crowley1992measuring}. This approach, however, is speculated on as the intrinsic value of products are subject to change. This fluctuation is a result of market driving forces. \cite{spangenberg1997measuring} argues that market analysis of hedonic versus utilitarian consumption scales are subject to the products being observed. These scales of consumer behaviour are based on the specific product categories analyzed, and are problematic when applied to different baskets of goods, or product categories. Although these scales are useful for case studies, they lack general applicability to products, services and other non-shopping activities \citep{spangenberg1997measuring}. A different approach to this problem would be to model consumer purchases over a set of topics which is based on product descriptions, rather than predefined scales. Thus, having a distribution of consumer purchases might be a useful scale to measure hedonic and utilitarian consumption patterns.

\

It is however important to fundamentally understand hedonic and utilitarian consumption. \cite{hirschman1982hedonic} theoretically defines hedonic consumption in the following way: "Hedonic consumption designates those facets of consumer behavior that relate the multi-sensory, fantasy and emotive aspects of one's experience with products". In other words, hedonic consumption can be seen as consumption based on sensory satisfaction, or excessive pleasure, which based on previous experiences. Moreover, \cite{okada2005justification} argues that hedonic consumption is related to a sensation of guilt. The author argues that hedonic behavior is associated with impulse purchases as those purchases are related to satisfying pleasurable sensations. In other words, hedonic consumption can be viewed as consumption based on pleasure. The notion is that consumers can purchase pleasure, and having experienced that pleasure one wants more of it. 

\

Measuring utilitarian consumption is considered to be the stark contrast of hedonic consumption. \cite{batra1991measuring} argues that utilitarian consumption is associated with the "expectation of consequences". The authors argue that utilitarian behaviour is associated with the outcome of purchasing products. With this, consumers are more diligent in their purchasing behaviour and consider the impact of their purchases \citep{batra1991measuring}. In other words, utilitarian consumption is associated with a rational thought process, where purchasing is associated with functionality of the products rather than the sensational experience of the products. For this reason, utilitarian consumption can also, in part, be considered as altruistic consumption. The reasoning for this is that consumers consider the impact of their purchases rather than just considering the pleasure derived from those purchases. 


\

The theoretical definitions of hedonic and utilitarian consumers are thus a sound definitions. However, empirically predicting whether consumers are hedonic or ultriustic poses a different challenge. As mentioned above, using product categories is subject to scrutiny when other products are considered, and thus empirical papers are case sensitive. The advent of big data might however bring about a new frontier when it comes to understanding the consumption patterns of individual consumers \cite{badea2014predicting}. Having a customer specific distributions over the products purchased might yield fruitful measurements, in the form of topics, for hedonic and utilitarian consumption. For this reason, machine learning methods can be applied to model these distributions. This paper thus aims at creating a distribution of consumers by using text mining algorithm called Latent Dirichlet Allocation. 

\

Most literature pertaining to marketing and LDA focus mainly on text rich sources such as social media and product reviews for the purposes of topic modeling. \cite{ma2013lda} uses LDA and synonym lexicon methods to extract product features from online costumer reviews of certain products. \cite{melville2009social} discuss techniques related to clustering social media discussions on products, and highlight the role of LDA in discovering topics of products based on blogers discussing products. Moreover, \cite{jacobs2016model} compares LDA and mixtures of Dirichlet-Multinomials (MDM) on online consumer purchasing data to predict which product a consumer would buy next. The authors note that LDA is more scalable relative to MDM, and would be useful on broader datasets containing purchasing information. 

\

The use of LDA in modeling consumer behaviour is nothing new. However, it's usage on purchase datasets is scarce. This is due to the nature of product descriptions. \cite{christidis2010exploring} explore this topic by using LDA to model consumer preferences as well as to effectively recommend products to consumers. In particular, the authors attempt to identify latent baskets and consumers from purchase data. They find intuitive baskets, modeled by their LDA structure, for different sets of products. However, a point to note is the lack of document information which is needed to distinguish topics from each other, and the authors suggest using social media and product hierarchies to further distinguish their topics \citep{christidis2010exploring}. This highlights the problem of using LDA on consumer purchases: Products have short descriptions. This leads to weak allocation of products to topics, as the LDA cannot distinguish between products. For this reason, the dataset used in this paper contains product descriptions augmented with Wikipedia definitions and descriptions of these products\footnote{See the data section of this paper for a description of the augmentation process}.   

\

As mentioned in the introduction, this paper has a twofold empirical purpose. The first is the constructing of consumer purchase distributions, and the second is to use these distributions to predict whether a consumer is of hedonic or utilitarian nature. Moreover, the second purpose has two components: Explaining and predicting hedonic behaviour by using the newly defined topics and other consumer demographic variables. The paper thus includes a logistic regression model for explanatory purposes, and a neural network and random forest for prediction. This means that a suitable predicted variable is necessary. The solution is partly based on marketing and economic literature, as well as intuition. However, this proved to be challenging given the scope of the dataset used in this analysis. Therefore, the predicted variable used for the models is whether a consumer purchased a gift or not; where purchasing a gift can be considered as a proxy for utilitarian behaviour and not purchasing a gift is considered as hedonic behaviour. The variable is subject to scrutiny, but it fits the scope of this analysis and is therefore used as the predicted variable. 

\

The intuition behind using gifts purchased as the predicted variable is quite simple. \cite{Sherry} states that gift giving is vehicle of social obligation, and is considered as charitable behaviour. The author explains that gift-giving behaviour is associated with a social contract, that of trust. The argument is that consumer give gifts to build social contracts, and that for that reason purchasing a particular gift is mostly premeditated. It is for this reason that the variable is a suitable proxy for utilitarian behaviour, as it not based on random excessive purchases (or hedonic consumption). Moreover, \cite{andreoni2002giving} explains that economists view gift giving as rational, altruistic, behaviour. This view is not based on opinion, but on theoretical economic principles such as game theory, whereby giving a gift is associated with trust building in sequential games. It is a form of utility maximization, outside of short term gain \citep{andreoni2002giving}. This is contrary to hedonic behaviour, and thus establishes a bases for the predicted variable used in this analysis. 


\


To summarize, this paper aims to predict whether a consumer is likely to purchase a gift by using the consumer's purchase history modeled over a set of topics and other demographic variables as predictor variables. The practical implication of the paper is a that of a recommendation to retailers and governments. It would be useful for producers, retailers and governments to know the distribution of their costumers' and constituents consumption behaviour. This method can be used to model customer preferences and decision behaviour. Costumers have different preferences and organization can utilize the tools presented in this paper to better understand their consumer base, which can be used to either implement social policies or individual marketing strategies. The recommendation is this: Datasets containing purchasing behaviour, such as the one used in this paper, should contain multidimensional product description. Possible ideas for variables defining the product are: A category variable for the product, a variable for compliment and substitute products, ingredients, production methods used describing the general recourses used in the industry producing the product. Such features describing the products would give a new dimension for researcher to use text mining algorithms to model consumer distributions among topics. Although ambitious, this recommendation is that organizations merely updated the text describing products, which will afford them a new avenue to model consumer behaviour. 


% section theoretical_background (end)